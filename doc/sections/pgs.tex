\section{Programming Grader Shell}
%\section{Python Grader Shell}
%\section{Personal Grader Shell}
\label{sec:pgs}

\tab As a partial solution to the CS TA grading problem, in Spring 2016
I wrote a simple program, Programming Grader Shell (PGS),
to automate and ease parts of the grading process.
Currently, the entire program consists of a single object-oriented
Python 3 script ($\sim 500$ lines) targeted at C/C++/Python programs.
The initial intention was to have a terminal-based tool
that could be easily ported and maintained.
As the semester progressed, I continued adding capabilities to PGS.
PGS runs from the command line and works by prompting questions
to the TA as the homeworks are traversed.
The TA selects the desired actions to proceed with the grading process.
Besides using PGS, the TA needs to record the grades, corrections, and feedback so
that these can be uploaded for the students to access. \par
\vspace{1em}

Prior to using PGS:
\begin{enumerate}
    \item Homework submissions have to reside in a single directory.
          The submission packages are assumed to be in archive or compressed formats
          (e.g., zip, tar, tgz, rar).
    \item Names of homework submissions should contain the student's unique key name. 
    \item A 3-column text file with each student's key, first, and last names.
\end{enumerate}

Using command line options, the TA specifies the homeworks' directory,
the text file with students' info, a working directory, and others. \par
\vspace{1em}

The current status of the program provides the following features:
\begin{itemize}
    \item Display list of students and homeworks
    \item Load set of input test files
    \item Unpack and traverse homework submission files
    \item Open files in a supported viewer
    \item Compile programs consisting of single or multiple source files and directories
    \item Run student's program
    \item Run student's program with selected input test file
    \item Begin grading from a specific student
\end{itemize}
\vspace{1em}

The following is a list of improvements that are being implemented for
the production version:
\begin{itemize}
    \item Support multiple programming languages
    \item Add mapping feature to load homework submissions directly from an output
          package provided by a course management software
    \item Run student's program using multiple input test files
    \item Record TA grades and comments
    \item Anonymity of students
    \item Randomization of the order student homeworks are processed
    \item Layer for interfacing with course management software or others
    \item Add option to choose either run versus recompile and run
\end{itemize}

\tab A problem that PGS does not solves automatically is that 
the set of input test files have to be generated by the TA.
Depending on the homework requirements, the number and order of inputs
may vary making necessary to even generate multiple versions of each test file.
My current approach consists of a combination of generating multiple versions
of each test file and reordering the student's code performing the input actions.
PGS makes this approach easy since source files can be opened, and
the program can be recompiled each time the TA wants to run it. \par

\tab In Spring 2017, I am TA again of \textit{``COSC 505''}, and had the idea of
redesigning PGS into a more versatile tool with a shell-like interface.
The shell would contain a state based on the student being evaluated and other factors.
The TA can perform a variety of actions using custom commands.
%As I understand the EECS department at UTK does not has a tool of its own to aid 
%teaching assistants during the grading process of graduate programming courses.
The PGS tool can serve as motivation for EECS department to make available 
a tool to alleviate the CS TA grading problem for those TAs that become interested.
Since this is a command line tool, I acknowledge that not necessarily
all TAs would make use of it.

